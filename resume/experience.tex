\cvsection{Experiencia Profesional}
\begin{cventries}
  \cventry
    {Maestra frente a grupo}
    {Colegio Zoltan Kodaly}
    {Morelia, Michoacán}
    {Ago 2020 - Jun 2021}
    {
      \begin{cvitems}
        \item {Tutora de planta de $3^{\circ}$ de primaria, con 25 alumnos a cargo.}
        \end{cvitems}
    }
    
    \cventry
    {Maestra frente a grupo}
    {Centro de Atención Múltiple 561}
    {Morelia, Michoacán}
    {Ago 2019 - Jun 2020}
    {
      \begin{cvitems}
        \item {Impartición de Español, Matemáticas y Ciencias a grupos de 10 alumnos; diseño de planes educativos especializados con material concreto acorde a la neurodidáctica.}
        \end{cvitems}
    }
    
    \cventry
    {Maestra de USAER}
    {Primaria ``Lázaro Cárdenas''}
    {Morelia, Michoacán}
    {Feb 2019 - Mar 2019}
    {
      \begin{cvitems}
        \item {Seguimiento de objetivos y resultados según calendario académico y recursos aprobados en proyectos con alumnado.}
        \end{cvitems}
    }
    
    \cventry
    {Maestra frente a grupo}
    {Centro de Atención Múltiple ``Lázaro Cárdenas''}
    {Pátzcuaro, Michoacán}
    {Sep 2018 - Oct 2018}
    {
      \begin{cvitems}
        \item {Dirección de grupo con barreras de aprendizaje, uso de lengua de signos mexicana y material didáctico adaptado.}
        \end{cvitems}
    }
    
    \cventry
    {Maestra frente a grupo}
    {Centro de Atención Múltiple 9}
    {Morelia, Michoacán}
    {Abr 2018 - May 2019}
    {
      \begin{cvitems}
        \item {Diseño de clases y material didáctico de Español y Matemáticas para alumnado con necesidades diversas.}
        \end{cvitems}
    }

    \cventry
    {Maestra frente a grupo}
    {Effeta y Universidad Anáhuac (FTA)}
    {Querétaro, Querétaro}
    {Ago 2023 - Abr 2025}
    {
      \begin{cvitems}
        \item {Docencia en grupo interdisciplinar con alumnado con discapacidad intelectual, auditiva, motriz y TEA, adaptando estrategias y ritmos de aprendizaje.}
        \item {Planificación de aprendizajes funcionales: matemáticas aplicadas, comprensión lectora, redacción sencilla, orientación temporal y conocimiento del entorno.}
        \item {Desarrollo de autonomía personal: higiene, hábitos de cuidado, organización del descanso y seguimiento de rutinas.}
        \item {Talleres de vida independiente (cocina y compras) y teatro/música para comunicación emocional, habilidades sociales y motrices.}
        \item {Apoyo en el programa FTA Anáhuac como guía sombra: acompañamiento en clases, actividades y socialización, con comunicación funcional en lengua de signos y salidas pedagógicas.}
        \item {Participación en proyectos de emprendimiento escolar: evaluación de productos, definición de precios y apoyo en procesos de venta.}
        \end{cvitems}
    }
  
 \end{cventries}
